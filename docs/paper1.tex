%%%%%%%%%%%%%%%%%%%%%%%%%%%%%%%%%%%%%%%%%%%%%%%%%%%%%%%%%%%%%%%%%%%%%%%%
%
% LensTractor Method Paper
%
%%%%%%%%%%%%%%%%%%%%%%%%%%%%%%%%%%%%%%%%%%%%%%%%%%%%%%%%%%%%%%%%%%%%%%%%

\documentclass[useAMS,usenatbib]{mn2e}
%% letterpaper
%% a4paper

% \voffset=-0.8in

% Packages:
% \input psfig.sty
\usepackage{xspace}
\usepackage{graphicx}

% Macros:
% JOURNALS
\newcommand {\apj} {ApJ}
\newcommand {\apjl} {ApJL}
\newcommand {\apjs} {ApJS}
\newcommand {\mnras} {MNRAS}
\newcommand {\apss} {Ap \& SS}
\newcommand {\aap} {A\&A}
\newcommand {\aj} {AJ}
\newcommand {\prd} {Phys. Rev. D}
\newcommand {\nat} {Nature}
\newcommand {\araa} {ARA\&A}
\newcommand {\jgr} {J. Geophys. Res.}
\newcommand {\pasp} {PASP}

% MISC
\newcommand {\etal} {et~al.~}
\def \spose#1{\hbox  to 0pt{#1\hss}}  
\newcommand {\lta} {\mathrel{\spose{\lower 3pt\hbox{$\sim$}}\raise  2.0pt\hbox{$<$}}}
\newcommand {\gta} {\mathrel{\spose{\lower  3pt\hbox{$\sim$}}\raise 2.0pt\hbox{$>$}}}
\def \ion#1#2{#1{\footnotesize{#2}}\relax}
\newcommand {\ha}  {\ifmmode H\alpha \else H$\alpha $ \fi} 
\newcommand {\hi} {\ion{H}{I} } 
% \newcommand {\Sref} {\S}
\def\Sref#1{Section~\ref{#1}\xspace}
\def\Fref#1{Figure~\ref{#1}\xspace}
\def\Tref#1{Table~\ref{#1}\xspace}
\def\Eref#1{Equation~\ref{#1}\xspace}
\def\Eqref#1{Eq.~(\ref{#1})\xspace}

% UNITS
\newcommand {\kms} {\ifmmode  \,\rm km\,s^{-1} \else $\,\rm km\,s^{-1}  $ \fi }
\newcommand {\kpc} {\ifmmode  {\rm kpc}  \else ${\rm  kpc}$ \fi  }  
\newcommand {\pc} {\ifmmode  {\rm pc}  \else ${\rm pc}$ \fi  }  
\newcommand {\Msun} {\ifmmode {\rm M_{\odot}} \else ${\rm M_{\odot}}$ \fi} 
\newcommand {\Zsun} {\ifmmode {\rm Z_{\odot}} \else ${\rm Z_{\odot}}$ \fi} 
\newcommand {\yr} {\ifmmode yr^{-1} \else $yr^{-1}$ \fi} 
\newcommand {\hMsun} {\ifmmode h^{-1}\,\rm M_{\odot} \else $h^{-1}\,\rm M_{\odot}$ \fi}

% COSMOLOGY
%\newcommand {\LCDM} {\ifmmode \Lambda{\rm CDM} \else $\Lambda{\rm CDM}$ \fi}

% LENSING
\def\zd{z_{\rm d}}
\def\zs{z_{\rm s}}
\def\zspdf{z_{\rm s,pdf}\;}
\def\Dd{D_{\rm d}}
\def\Ds{D_{\rm s}}
\def\Dds{D_{\rm ds}}
\def\Sigmacrit{\Sigma_{\rm crit}}
\def\REin{R_{\rm Ein}}
\def\MEin{M_{\rm Ein}}
\def\sigmasie{\sigma_{\mathrm{SIE}}}
\def\Mvir{M_{\rm vir}}
\def\Mhalo{M_{\rm h}}
\def\Vhalo{V_{\rm c,h}}
\def\rhalo{r_{\rm c,h}}
\def\qhalo{q_{3,\rm h}}
\def\vc{V_{\rm c}}
\def\rc{r_{\rm c}}
\def\q3{q_{3}}
\def\bsis{b_{\rm SIS}}

% SED/PHOTOMETRIC FITTING
\def\Mstar{M_{*}}
\def\logMstar{\log_{10}\left(\Mstar/\Msun\right)}
\def\Mstarb{M_{*,\rm b}}
\def\logMstarb{\log_{10}\left(\Mstarb/\Msun\right)}
\def\Mstard{M_{*,\rm d}}
\def\logMstard{\log_{10}\left(\Mstard/\Msun\right)}
% \def\Reff{R_{\rm eff}}
% \def\Reffb{R_{\rm eff,b}}
% \def\Reffd{R_{\rm eff,d}}
\def\Reff{R_{\rm 50}}
\def\Reffb{R_{\rm 50,b}}
\def\Reffd{R_{\rm 50,d}}
\def\sersic{S\'ersic}
\def\nb{n_{\rm b}}
\def\qb{q_{\rm b}}
\def\phib{phi_{\rm b}}
\def\nd{n_{\rm d}}
\def\qd{q_{\rm d}}
\def\phid{phi_{\rm d}}

% SOFTWARE/HARDWARE
\def\SExtractor{{\sc SExtractor}\xspace}
\def\hst{{\it HST}\xspace}
\def\acs{\hst/ACS\xspace}
\def\galfit{{\sc galfit}\xspace}
\def\idl{{\sc idl}\xspace}
\def\python{{\sc python}\xspace}
\def\SPASMOID{{\sc SPASMOID}\xspace}

% FILTERS
\def\Bfilter{F450W\xspace}
\def\Vfilter{F606W\xspace}
\def\Ifilter{F814W\xspace}
\def\Kfilter{K'\xspace}
\def\Bband{\Bfilter-band\xspace}
\def\Vband{\Vfilter-band\xspace}
\def\Iband{\Ifilter-band\xspace}
\def\Kband{\Kfilter-band\xspace}

% PROBABILITY THEORY
\def\pr{{\rm Pr}}
\def\data{{\mathbf{d}}}
\def\datap{{\mathbf{d}^{\rm p}}}
\def\datai{d_i}
\def\datapi{d^{\rm p}_i}
\def\masspars{\boldsymbol{\theta}_{\rm m}}
\def\srcpars{\boldsymbol{\theta}_{\rm s}}
\def\vrot{{\mathbf{v}}}
\def\vrotp{{\mathbf{v}^{\rm p}}}
\def\vrotmodel{\hat{\mathbf{v}}}
\def\vrotj{v_j}
\def\vrotpj{v^{\rm p}_j}

% SPECTROSCOPY
\def\Angstrom{A\xspace}
\def\CaII{Ca\,{\sc ii}\xspace}
\def\NaII{Na\,{\sc ii}\xspace}
\def\NaD{Na\,{\sc D}\xspace} % ??
\def\NII{N\,{\sc ii}\xspace}
\def\HeII{He\,{\sc ii}\xspace}
%\def\Mgb{Mg\,{\sc ii}} Mg II is something else, at 2800 A no?
\def\Mgb{Mg\,{\rm b}\xspace}
\def\Ha{H$\alpha$\xspace}
\def\Hb{H$\beta$\xspace}
\def\OII{[O\,{\sc ii}]\xspace}
\def\OIII{O\,{\sc iii}]\xspace}
\def\CIII{C\,{\sc iii}]\xspace}
\def\CIV{C\,{\sc iv}\xspace}
\def\FeII{[Fe\,{\sc ii}]\xspace}
\def\sigmap{$\sigma_{\rm ap}$}
\def\sigmasdss{$\sigma_{\rm SDSS}$}

% PAPER SERIES
\def\paperI{{Paper~I}\xspace}
\def\paperIfirsttime{{Treu \etal submitted, referred to hereafter as \paperI}\xspace}
\def\paperII{{Paper~II}\xspace}
\def\paperIIfirsttime{{Dutton \etal submitted, referred to hereafter as \paperII}\xspace}
\def\paperIII{{Paper~III}\xspace}
\def\paperIIIfirsttime{{Brewer \etal submitted, referred to hereafter as \paperIII}\xspace}

% SAMPLE PROPERTIES
\def\NSWELLS{27}
\def\NSWELLSNEW{19}
\def\NSWELLSHST{16}
\def\NSWELLSAO{3}
\def\NSWELLSA{8}
\def\NSWELLSB{1}
\def\NSWELLSC{6}
\def\NSWELLSX{4}
%10 spirals from slacs, including 8 ``edge-on''
\def\NSSLACS{10}
\def\NSESLACS{8}
\def\NTOTALA{16}

% COMMENTING
\usepackage[usenames]{color}
\newcommand{\aaron}[1]{\textcolor{Brown}{\bf #1}}
\newcommand{\phil}[1]{\textcolor{blue}{\bf #1}}
\newcommand{\brendon}[1]{\textcolor{Violet}{\bf #1}}
\newcommand{\tommaso}[1]{\textcolor{green}{\bf #1}}
\newcommand{\matt}[1]{\textcolor{orange}{\bf #1}}
\newcommand{\matteo}[1]{\textcolor{Magenta}{\bf #1}}
\newcommand{\flag}[1]{\textcolor{red}{\bf #1}}
\newcommand{\achtung}[2]{\textcolor{red}{\it\bf ATTENTION #1! #2}}

\def\kipac{Kavli Institute for Particle Astrophysics and Cosmology, Stanford University, 452 Lomita Mall, Stanford, CA 94035, USA}
\def\oxford{Department of Physics, University of Oxford, Keble Road, Oxford, OX1 3RH, UK}
\def\nyu{NYU}
\def\cmu{CMU}
\def\mpia{MPIA}
\def\ucsb{UCSB}

\def\pjmemail{\tt dr.phil.marshall@gmail.com}


%%%%%%% RESULTS %%%%%%%%%

%%%%%%%%%%%%%%%%%%%%%%%%%

%%%%%%%%%%%%%%%%%%%%%%%%%%%%%%%%%%%%%%%%%%%%%%%%%%%%%%%%%%%%%%%%%%%%%%%%

\title[Finding lensed quasars]
{Detecting strongly lensed quasars by pixel-level model comparison}

\author[]{%
  Philip J. Marshall$^{1,2}$\thanks{\pjmemail},
  David W. Hogg$^{3,4}$,
  Adriano Agnello$^{5}$,
  Dustin Lang$^{6}$,
  Eric P. Morganson$^{4}$,
  and others 
  \medskip\\
  $^1$\kipac\\
  $^2$\oxford\\
  $^3$\nyu\\
  $^4$\mpia\\
  $^4$\ucsb\\
  $^6$\cmu
}

%%%%%%%%%%%%%%%%%%%%%%%%%%%%%%%%%%%%%%%%%%%%%%%%%%%%%%%%%%%%%%%%%%%%%%%%

\begin{document}
             
\date{to be submitted to MNRAS}
             
\pagerange{\pageref{firstpage}--\pageref{lastpage}}\pubyear{2010}

\maketitle           

\label{firstpage}

%%%%%%%%%%%%%%%%%%%%%%%%%%%%%%%%%%%%%%%%%%%%%%%%%%%%%%%%%%%%%%%%%%%%%%%%

\begin{abstract}
In some heuristic sense, we will end up believing that a particular
astronomical source is a galaxy-scale strong gravitational lens only if it
is more plausible to explain it as a superposition of a gravitationally
distorted background source and a foreground lensing galaxy than it is to
explain it as a single object with very odd morphology. The automated
detection of strong gravitational lenses must involve consideration of
these plausbilities.
%
We investigate the explicit implementation of this principle in practice,
in the context of automated searches for lensed quasars in optical imaging
surveys. We construct a quantitative approximation to this comparison of
plausibilities by fitting multi-epoch, multi-band ground-based imaging of
gravitational lens candidates with two image models, one of which,
``Lens,'' is a point source being lensed by a simple model massive galaxy,
and the other is a complex ``Nebula'' constructed from a simple galaxy and
$K$ point sources.  We \emph{approximate} the relative plausibility of the
lensing hypothesis with a likelihood ratio between these models, evaluated
at best-fit (that is, not marginalized), penalized by a simple factor
accounting for relative model freedom.
%
We fit these models and compute the relative lensing plausibility for 10
confirmed lensed quasars and 40 confirmed non-lenses from the SDSS Quasar
Lens Search (SQLS).  We find that we [can/cannot] confidently identify the
lenses and rule out the non-lenses with our plausibility criteria.  Even
though the plausibility calculation involves fitting highly parametrized
models of astronomical sources and the varying point-spread function in
many pixels of imaging, it can be calculated in seconds to minutes on
contemporary hardware; we comment on how this scales to ongoing lens
searches in various wide field imaging surveys.
\end{abstract}

% Full list of options at http://www.journals.uchicago.edu/ApJ/instruct.key.html

\begin{keywords}
  gravitational lensing
\end{keywords}

\setcounter{footnote}{1}

%%%%%%%%%%%%%%%%%%%%%%%%%%%%%%%%%%%%%%%%%%%%%%%%%%%%%%%%%%%%%%%%%%%%%%%%
%%  SECTION 1: INTRODUCTION
%%%%%%%%%%%%%%%%%%%%%%%%%%%%%%%%%%%%%%%%%%%%%%%%%%%%%%%%%%%%%%%%%%%%%%%%

\section{Introduction}
\label{sec:intro}

Strong gravitational lenses are useful. Lensed quasars are useful for time
delay lens cosmography, accretion disk studies, as well as investigations
of the galaxy mass distributions (including dark matter density profiles
and subhalo populations). All strong lensing science projects would benefit
from larger samples of lenses. 

Wide-field imaging surveys are good sources of new lenses: for example 
lensed quasars are expected to occur at a frequency of 1 in $10^{3-4}$ or
so, equivalent to $\sim0.1$ per square degree (depending on survey image
depth). The corresponding numbers of objects of any type is $\sim10^{4-5}$
per square degree. How are we going to achieve high efficiency, in the
absence of additional information? Catalog searching will be essential in
the first instance.  However, most survey catalogs will be general purpose,
and so may well not be optimized for revealing strong lenses. The features
of a galaxy-scale strong lens have typical  separations comparable to the
image quality, so that the information content of the catalogs will be
limited by the object deblender used in their construction. Eventually the
image pixels will need to be examined using specialist lens detection
tools, on as small a sample of ``targets'' as possible.

Lens candidacy requires the targets to be modeled, in order to quantify the
plausibility of the lensing hypothesis. In the cases where the features are
resolved, the target's catalog data will be able to be  assessed in the
context of a lens model. (This will also be possible when using catalogs of
unresolved  objects, albeit at lower fidelity.) Certainly when moving to
analysis of the image pixels we are on familiar ground: lens models that
predict image pixels are well established in the literature (see work by
Koopmans, Marshall, Suyu, Auger, Vegetti, Barnabe et al).  While not
usually described this way, visual inspection involves mental modeling of
images. The Space Warps project is investigating this technique at large
scale, following Faure, Jackson et al. The alternative is robotic modeling,
as in HAGGLeS. Which is implemented earlier in the pipeline, robotic
modeling or human inspection, is a matter for investigation.

Two possible routes to quantifying lensing plausibility are as follows. One
is just to consider the probability of the data given the hypothesis, and
carry out something like a chi-squared goodness of fit test to approximate
this.  In practice, we do not expect simple models (that can be computed at
low computational cost) to fit the data well, and so classifying targets
based on goodness of fit alone is likely to need to be calibrated, using a
training set of lenses and non-lenses. A second approach is to consider,
and compute, the probability of the data given two competing hypotheses,
`lens'' and ``non-lens,'' and compare them, bypassing the need to calibrate
the goodness of fit. We might expect this approach to be more automatable,
provided we set up the models to have comparable fitting power in the case
where the system really is a lens.  Then, the lens model would be
naturally favoured, since it's parameter space is lower-dimensional, and
its prior PDFs more constraining: lens models are more efficient
descriptions of data generated by gravitational lenses. In the many cases
where the system is really a non-lens, the probability of the data given
the non-lens model should be higher: the non-lens model should provide a
higher goodness of fit when the data was not generated by a gravitational
lens.

Let us suppose that some combination of catalog target selection and
preliminary image inspection or analysis has been performed, leaving us
with a large sample of prime targets, and reasonable catalogs of the
features in their images. Our task is then to extract the remaining
information from the data, by modeling the images and comparing them,
probabilistically.  In this paper, we introduce two simple models for
generating the predicted  images, and an approximation to the probablilty
of the data given each one.

We then ask the following questions:
\begin{enumerate}
%
\item Can a reliable relative lens plausibility be calculated based on
approximate probabilistic model comparison at the pixel level?
%
\item Can such a plausibility ratio be used to distinguish lenses from
non-lenses across a realistic sample of pre-selected targets, such as those
selected from the SDSS spectroscopic object catalog during the SQLS?
%
\item What challenges will be faced when attempting a similar computation in a
wide-field imaging survey covering thousands of square degrees and potentially
millions of targets? And with which further approximations might these be
approached?
%
\end{enumerate}

Wide field optical imaging survey data have, generically, the  property
that the image quality varies from epoch to epoch, and so between images
taken in different filters. This results from the need to cover large areas
of sky at low overhead. An advantage of this observing strategy is that the
multiplicity of images allows missing data or poor image quality in any
given single exposure to be corrected for by the other exposures. To
capitalise on this, and to extract as much information as possible from the
data, we propose a joint fit of the same model to all images
simultaneously. We do this using the flexible image fitting software ``The
Tractor'' (Lang \& Hogg 2014). This approach is optimal in terms of its
potential for information extraction, but its cost is to be investigated.

An alternative approach is explored in a paper by Chan et al (2014). These
authors investigate the greater execution speed achievable by first
extracting point source positions by ad-hoc means, and then fitting lens
models to these reduced data instead of the pixel values. They also pursue
the first method for assessing lens plausibility, focusing on the goodness
of fit of the lens models, and interpreting this in the context of a
training set. Comparison between the two approaches will help guide the
community as it approaches the wide field survey datasets of DES, HSC and
LSST.

This paper is organised as follows. In \Sref{sec:model} we describe our two
competing models, and the approximate relative plausibility[ies?] we
propose. These are implemented in a piece of software called \LT, which we
also briefly describe. In \Sref{sec:data} we define an example dataset,
and a test dataset, which we use to investigate our proposed approach to
lens classification in \Sref{sec:results}. We discuss the implications of
our results in \Sref{sec:discuss}, and draw brief conclusions in
\Sref{sec:concl}.


%%%%%%%%%%%%%%%%%%%%%%%%%%%%%%%%%%%%%%%%%%%%%%%%%%%%%%%%%%%%%%%%%%%%%%%%
%%  SECTION 2: MODELS AND INFERENCE
%%%%%%%%%%%%%%%%%%%%%%%%%%%%%%%%%%%%%%%%%%%%%%%%%%%%%%%%%%%%%%%%%%%%%%%%

\section{Models and Inference}
\label{sec:model}

\subsection{Models: Lenses and Nebulae}
\label{sec:model:ln}

\subsection{Inference with The Tractor}
\label{sec:model:tractor}

\subsection{Model Comparison by Approximate Plausibility Ratio}
\label{sec:model:bic}


%%%%%%%%%%%%%%%%%%%%%%%%%%%%%%%%%%%%%%%%%%%%%%%%%%%%%%%%%%%%%%%%%%%%%%%%
%%  SECTION 3: DATA
%%%%%%%%%%%%%%%%%%%%%%%%%%%%%%%%%%%%%%%%%%%%%%%%%%%%%%%%%%%%%%%%%%%%%%%%

\section{Data}
\label{sed:data}


\subsection{Example}
\label{sec:data:example}

As a simple first example, we consider the lensed quasar system H1413$+$117
\citep{H1413}, as observed three times during the PS1 $3\pi$ survey. We 
made 10-arcsecond postage stamp images from three frames taken over the
course of one year (MJDs  55377, 55664, 55755) in two filters ($r$ and
$z$). The image quality varies considerably between the three observing
epochs.

% \section{Test Dataset}
% \label{sec:data:test}


%%%%%%%%%%%%%%%%%%%%%%%%%%%%%%%%%%%%%%%%%%%%%%%%%%%%%%%%%%%%%%%%%%%%%%%%
%%  SECTION 4: RESULTS
%%%%%%%%%%%%%%%%%%%%%%%%%%%%%%%%%%%%%%%%%%%%%%%%%%%%%%%%%%%%%%%%%%%%%%%%

\section{Results}
\label{sec:results}

\subsection{Example}
\label{sec:results:example}

We first illustrate the operation of \LT with a sequence of plots generated
by the code. In \Fref{fig:H1413example-progress}, we show the Nebula4 model
fit just before the PSF parameters were unfrozen, in order to show the
impact of assuming a PSF model just based on the FWHM keyword in the image
headers. The model was initialized by hand, using ds9 to identify the image
positions.

%%%%%%%%%%%%%%%%%%%%%%%%%%
\begin{figure*}
\centerline{
\includegraphics[width=0.9\linewidth]{figs/H1413+117_10x10arcsec_progress-06_optimizing_Nebula4.png}}
\caption{\LT example: three images of the known lens H1413+117, analyzed using
the Nebula4 model. This snapshot was taken mid-way through the inference, just
before the PSF parameters were unfrozen.}
\label{fig:H1413example-progress}
\end{figure*}
%%%%%%%%%%%%%%%%%%%%%%%%%%

\Fref{fig:H1413example-final} shows the final result of this fit, with both
the Nebula4 and PSF models optimized. The residuals are small but not
consistent with pure noise, suggesting either that the fit has not yet
converged, or that our simple model is not able to capture all the details
in the data, or both.

\action{PJM}{Add cartoon model overlays so we can check the fits.}

%%%%%%%%%%%%%%%%%%%%%%%%%%
\begin{figure*}
\centerline{
\includegraphics[width=0.9\linewidth]{figs/H1413+117_10x10arcsec_progress-09_optimizing_Nebula4.png}}
\caption{\LT example: the final result from the inference shown in
\Fref{fig:H1413example-progress}, with both the object and PSF models
optimized.}
\label{fig:H1413example-final}
\end{figure*}
%%%%%%%%%%%%%%%%%%%%%%%%%%

The performance of the Lens model initialization is quite sensitive to its
initialization. Ad-hoc placement of the model sources behind the 
model lens does not work reliably. Instead, we map the Nebula4 point source
positions back to the source plane through a lens with Einstein radius
estimated from the average separation between the Nebula4 point sources and
their mean position, and adjust the shear until the four corresponding
posint source positions match as closely as possible. Such a ``source plane
minimisation'' is fast, and accurate enough to initialize the lens model
close to the peak of its likelihood function. 

\comment{PJM}{This is a hypothesis, to be tested.}

\action{PJM}{Implement this Lens intialization.}


%%%%%%%%%%%%%%%%%%%%%%%%%%%%%%%%%%%%%%%%%%%%%%%%%%%%%%%%%%%%%%%%%%%%%%%%
%%  SECTION 5: DISCUSSION
%%%%%%%%%%%%%%%%%%%%%%%%%%%%%%%%%%%%%%%%%%%%%%%%%%%%%%%%%%%%%%%%%%%%%%%%

\section{Discussion}
\label{sec:discuss}

%%%%%%%%%%%%%%%%%%%%%%%%%%%%%%%%%%%%%%%%%%%%%%%%%%%%%%%%%%%%%%%%%%%%%%%%
%%  SECTION 6: CONCLUSIONS
%%%%%%%%%%%%%%%%%%%%%%%%%%%%%%%%%%%%%%%%%%%%%%%%%%%%%%%%%%%%%%%%%%%%%%%%

\section{Conclusions}
\label{sec:concl}

Summary.

Our main results can be summarised as follows:

\begin{enumerate}

\item Effectiveness

\item Feasibility

\end{enumerate}

Future challenges.

%%%%%%%%%%%%%%%%%%%%%%%%%%%%%%%%%%%%%%%%%%%%%%%%%%%%%%%%%%%%%%%%%%%%%%%%
%%  ACKNOWLEDGMENTS
%%%%%%%%%%%%%%%%%%%%%%%%%%%%%%%%%%%%%%%%%%%%%%%%%%%%%%%%%%%%%%%%%%%%%%%%

\section*{Acknowledgments}
 
We thank Sherry Suyu and James H.\ H.\ Chan, and Matt Auger and the STRIDES
collaboration for useful discussions and suggestions.
PJM acknowledges support from grants\ldots
DWH acknowledges support from grants\ldots

PJM was given support by the Royal 
Society, in the form of a research fellowship.
%
DWH ...
%
DL ...
%
% PS1



%%%%%%%%%%%%%%%%%%%%%%%%%%%%%%%%%%%%%%%%%%%%%%%%%%%%%%%%%%%%%%%%%%%%%%%%
%%  REFERENCES
%%%%%%%%%%%%%%%%%%%%%%%%%%%%%%%%%%%%%%%%%%%%%%%%%%%%%%%%%%%%%%%%%%%%%%%%

% MNRAS does not use bibtex, input .bbl file instead. 
% Generate this in the makefile using bubble script in scriptutils:

%   bubble -f swells-survey.tex references.bib 

\bibliographystyle{apj}
\bibliography{references}

%%%%%%%%%%%%%%%%%%%%%%%%%%%%%%%%%%%%%%%%%%%%%%%%%%%%%%%%%%%%%%%%%%%%%%%%

\label{lastpage}
\bsp

\end{document}

%%%%%%%%%%%%%%%%%%%%%%%%%%%%%%%%%%%%%%%%%%%%%%%%%%%%%%%%%%%%%%%%%%%%%%%%
