%%%%%%%%%%%%%%%%%%%%%%%%%%%%%%%%%%%%%%%%%%%%%%%%%%%%%%%%%%%%%%%%%%%%%%%%
%
% LensTractor Method Paper
%
%%%%%%%%%%%%%%%%%%%%%%%%%%%%%%%%%%%%%%%%%%%%%%%%%%%%%%%%%%%%%%%%%%%%%%%%

\documentclass[useAMS,usenatbib]{mn2e}
%% letterpaper
%% a4paper

% \voffset=-0.8in

% Packages:
% \input psfig.sty
\usepackage{xspace}
\usepackage{graphicx}

% Macros:
% Project stuff:
\def\panstarrs{{\sc PS1}\xspace}

% JOURNALS
\newcommand {\apj} {ApJ}
\newcommand {\apjl} {ApJL}
\newcommand {\apjs} {ApJS}
\newcommand {\mnras} {MNRAS}
\newcommand {\apss} {Ap \& SS}
\newcommand {\aap} {A\&A}
\newcommand {\aj} {AJ}
\newcommand {\prd} {Phys. Rev. D}
\newcommand {\nat} {Nature}
\newcommand {\araa} {ARA\&A}
\newcommand {\jgr} {J. Geophys. Res.}
\newcommand {\pasp} {PASP}

% MISC
\newcommand {\etal} {et~al.~}
\def \spose#1{\hbox  to 0pt{#1\hss}}  
\newcommand {\lta} {\mathrel{\spose{\lower 3pt\hbox{$\sim$}}\raise  2.0pt\hbox{$<$}}}
\newcommand {\gta} {\mathrel{\spose{\lower  3pt\hbox{$\sim$}}\raise 2.0pt\hbox{$>$}}}
\def \ion#1#2{#1{\footnotesize{#2}}\relax}
\newcommand {\ha}  {\ifmmode H\alpha \else H$\alpha $ \fi} 
\newcommand {\hi} {\ion{H}{I} } 
% \newcommand {\Sref} {\S}
\def\Sref#1{Section~\ref{#1}\xspace}
\def\Fref#1{Figure~\ref{#1}\xspace}
\def\Tref#1{Table~\ref{#1}\xspace}
\def\Eref#1{Equation~\ref{#1}\xspace}
\def\Eqref#1{Eq.~(\ref{#1})\xspace}

% UNITS
\newcommand {\kms} {\ifmmode  \,\rm km\,s^{-1} \else $\,\rm km\,s^{-1}  $ \fi }
\newcommand {\kpc} {\ifmmode  {\rm kpc}  \else ${\rm  kpc}$ \fi  }  
\newcommand {\pc} {\ifmmode  {\rm pc}  \else ${\rm pc}$ \fi  }  
\newcommand {\Msun} {\ifmmode {\rm M_{\odot}} \else ${\rm M_{\odot}}$ \fi} 
\newcommand {\Zsun} {\ifmmode {\rm Z_{\odot}} \else ${\rm Z_{\odot}}$ \fi} 
\newcommand {\yr} {\ifmmode yr^{-1} \else $yr^{-1}$ \fi} 
\newcommand {\hMsun} {\ifmmode h^{-1}\,\rm M_{\odot} \else $h^{-1}\,\rm M_{\odot}$ \fi}

% COSMOLOGY
%\newcommand {\LCDM} {\ifmmode \Lambda{\rm CDM} \else $\Lambda{\rm CDM}$ \fi}

% LENSING
\def\zd{z_{\rm d}}
\def\zs{z_{\rm s}}
\def\zspdf{z_{\rm s,pdf}\;}
\def\Dd{D_{\rm d}}
\def\Ds{D_{\rm s}}
\def\Dds{D_{\rm ds}}
\def\Sigmacrit{\Sigma_{\rm crit}}
\def\REin{R_{\rm Ein}}
\def\MEin{M_{\rm Ein}}
\def\sigmasie{\sigma_{\mathrm{SIE}}}
\def\Mvir{M_{\rm vir}}
\def\Mhalo{M_{\rm h}}
\def\Vhalo{V_{\rm c,h}}
\def\rhalo{r_{\rm c,h}}
\def\qhalo{q_{3,\rm h}}
\def\vc{V_{\rm c}}
\def\rc{r_{\rm c}}
\def\q3{q_{3}}
\def\bsis{b_{\rm SIS}}

% SED/PHOTOMETRIC FITTING
\def\Mstar{M_{*}}
\def\logMstar{\log_{10}\left(\Mstar/\Msun\right)}
\def\Mstarb{M_{*,\rm b}}
\def\logMstarb{\log_{10}\left(\Mstarb/\Msun\right)}
\def\Mstard{M_{*,\rm d}}
\def\logMstard{\log_{10}\left(\Mstard/\Msun\right)}
% \def\Reff{R_{\rm eff}}
% \def\Reffb{R_{\rm eff,b}}
% \def\Reffd{R_{\rm eff,d}}
\def\Reff{R_{\rm 50}}
\def\Reffb{R_{\rm 50,b}}
\def\Reffd{R_{\rm 50,d}}
\def\sersic{S\'ersic}
\def\nb{n_{\rm b}}
\def\qb{q_{\rm b}}
\def\phib{phi_{\rm b}}
\def\nd{n_{\rm d}}
\def\qd{q_{\rm d}}
\def\phid{phi_{\rm d}}

% SOFTWARE/HARDWARE
\def\SExtractor{{\sc SExtractor}\xspace}
\def\hst{{\it HST}\xspace}
\def\acs{\hst/ACS\xspace}
\def\galfit{{\sc galfit}\xspace}
\def\idl{{\sc idl}\xspace}
\def\python{{\sc python}\xspace}
\def\SPASMOID{{\sc SPASMOID}\xspace}

% FILTERS
\def\Bfilter{F450W\xspace}
\def\Vfilter{F606W\xspace}
\def\Ifilter{F814W\xspace}
\def\Kfilter{K'\xspace}
\def\Bband{\Bfilter-band\xspace}
\def\Vband{\Vfilter-band\xspace}
\def\Iband{\Ifilter-band\xspace}
\def\Kband{\Kfilter-band\xspace}

% PROBABILITY THEORY
\def\pr{{\rm Pr}}
\def\data{{\mathbf{d}}}
\def\datap{{\mathbf{d}^{\rm p}}}
\def\datai{d_i}
\def\datapi{d^{\rm p}_i}
\def\masspars{\boldsymbol{\theta}_{\rm m}}
\def\srcpars{\boldsymbol{\theta}_{\rm s}}
\def\vrot{{\mathbf{v}}}
\def\vrotp{{\mathbf{v}^{\rm p}}}
\def\vrotmodel{\hat{\mathbf{v}}}
\def\vrotj{v_j}
\def\vrotpj{v^{\rm p}_j}

% SPECTROSCOPY
\def\Angstrom{A\xspace}
\def\CaII{Ca\,{\sc ii}\xspace}
\def\NaII{Na\,{\sc ii}\xspace}
\def\NaD{Na\,{\sc D}\xspace} % ??
\def\NII{N\,{\sc ii}\xspace}
\def\HeII{He\,{\sc ii}\xspace}
%\def\Mgb{Mg\,{\sc ii}} Mg II is something else, at 2800 A no?
\def\Mgb{Mg\,{\rm b}\xspace}
\def\Ha{H$\alpha$\xspace}
\def\Hb{H$\beta$\xspace}
\def\OII{[O\,{\sc ii}]\xspace}
\def\OIII{O\,{\sc iii}]\xspace}
\def\CIII{C\,{\sc iii}]\xspace}
\def\CIV{C\,{\sc iv}\xspace}
\def\FeII{[Fe\,{\sc ii}]\xspace}
\def\sigmap{$\sigma_{\rm ap}$}
\def\sigmasdss{$\sigma_{\rm SDSS}$}

% PAPER SERIES
\def\paperI{{Paper~I}\xspace}
\def\paperIfirsttime{{Treu \etal submitted, referred to hereafter as \paperI}\xspace}
\def\paperII{{Paper~II}\xspace}
\def\paperIIfirsttime{{Dutton \etal submitted, referred to hereafter as \paperII}\xspace}
\def\paperIII{{Paper~III}\xspace}
\def\paperIIIfirsttime{{Brewer \etal submitted, referred to hereafter as \paperIII}\xspace}

% SAMPLE PROPERTIES
\def\NSWELLS{27}
\def\NSWELLSNEW{19}
\def\NSWELLSHST{16}
\def\NSWELLSAO{3}
\def\NSWELLSA{8}
\def\NSWELLSB{1}
\def\NSWELLSC{6}
\def\NSWELLSX{4}
%10 spirals from slacs, including 8 ``edge-on''
\def\NSSLACS{10}
\def\NSESLACS{8}
\def\NTOTALA{16}

% COMMENTING
\usepackage[usenames]{color}
\newcommand{\aaron}[1]{\textcolor{Brown}{\bf #1}}
\newcommand{\phil}[1]{\textcolor{blue}{\bf #1}}
\newcommand{\brendon}[1]{\textcolor{Violet}{\bf #1}}
\newcommand{\tommaso}[1]{\textcolor{green}{\bf #1}}
\newcommand{\matt}[1]{\textcolor{orange}{\bf #1}}
\newcommand{\matteo}[1]{\textcolor{Magenta}{\bf #1}}
\newcommand{\flag}[1]{\textcolor{red}{\bf #1}}
\newcommand{\achtung}[2]{\textcolor{red}{\it\bf ATTENTION #1! #2}}

\def\kipac{Kavli Institute for Particle Astrophysics and Cosmology, Stanford University, 452 Lomita Mall, Stanford, CA 94035, USA}
\def\oxford{Department of Physics, University of Oxford, Keble Road, Oxford, OX1 3RH, UK}
\def\nyu{NYU}
\def\cmu{CMU}
\def\mpia{MPIA}
\def\ucsb{UCSB}

\def\pjmemail{\tt dr.phil.marshall@gmail.com}


%%%%%%% RESULTS %%%%%%%%%

%%%%%%%%%%%%%%%%%%%%%%%%%

%%%%%%%%%%%%%%%%%%%%%%%%%%%%%%%%%%%%%%%%%%%%%%%%%%%%%%%%%%%%%%%%%%%%%%%%

\title[Finding lensed quasars]
{A simple probabilistic approach to finding 
strongly lensed quasars in ground-based imaging}

\author[]{%
  Philip J. Marshall$^{1,2}$\thanks{\pjmemail},
  David W. Hogg$^{3,4}$,
  Adriano Agnello$^{5}$,
  Dustin Lang$^{6}$,
  Eric P. Morganson$^{4}$,
  and others 
  \medskip\\
  $^1$\kipac\\
  $^2$\oxford\\
  $^3$\nyu\\
  $^4$\mpia\\
  $^4$\ucsb\\
  $^6$\cmu
}

%%%%%%%%%%%%%%%%%%%%%%%%%%%%%%%%%%%%%%%%%%%%%%%%%%%%%%%%%%%%%%%%%%%%%%%%

\begin{document}
             
\date{to be submitted to MNRAS}
             
\pagerange{\pageref{firstpage}--\pageref{lastpage}}\pubyear{2010}

\maketitle           

\label{firstpage}

%%%%%%%%%%%%%%%%%%%%%%%%%%%%%%%%%%%%%%%%%%%%%%%%%%%%%%%%%%%%%%%%%%%%%%%%

\begin{abstract}
In some heuristic sense, we end up believing that a particular source
is a galaxy-scale strong gravitational lens because it is more
plausible to explain it as a superposition of a gravitationally
distorted background source and a foreground lensing galaxy than it is
to explain it as a single object with very odd morphology.  Here we
construct a quantitative approximation to this comparison of
plausibilities by fitting multi-epoch, multi-band ground-based imaging
of gravitational lens candidates with two image models, one of which, ``Lens,''
is a point source being lensed by a simple model massive galaxy, and the
other is a complex ``Nebula'' constructed from a simple galaxy and $K$
point sources.  The relative plausibility of the lensing hypothesis is
\emph{approximated} by a likelihood ratio between these models, evaluated at
best-fit (that is, not marginalized), penalized by a factor accounting
for relative model freedom.  We fit these models and compute lensing
plausibility for 
% four candidate lens sources with \panstarrs\ $grizY$ imaging.  
10 confirmed lensed quasars and 40 confirmed non-lenses from the SDSS Quasar Lens
Search (SQLS).
% Two of the sources are known lenses, and two are not;
We find that we can confidently identify the lenses and rule out the
non-lenses with our plausibility criteria.  Even though the plausibility
calculation involves fitting highly parametrized models of
astronomical sources and the varying point-spread function in many
pixels of imaging, it can be calculated in 
% seconds
a few minutes on contemporary
hardware; we comment on how this scales to ongoing lens searches in various wide
field imaging surveys.
\end{abstract}

% Full list of options at http://www.journals.uchicago.edu/ApJ/instruct.key.html

\begin{keywords}
  gravitational lensing
\end{keywords}

\setcounter{footnote}{1}

%%%%%%%%%%%%%%%%%%%%%%%%%%%%%%%%%%%%%%%%%%%%%%%%%%%%%%%%%%%%%%%%%%%%%%%%
%%  SECTION 1: INTRODUCTION
%%%%%%%%%%%%%%%%%%%%%%%%%%%%%%%%%%%%%%%%%%%%%%%%%%%%%%%%%%%%%%%%%%%%%%%%

\section{Introduction}
\label{sec:intro}

In this paper we ask the following questions:

\begin{enumerate}

\item Can a relative lens plausibility be reliably computed across a
realistic sample of pre-selected targets, such as those selected from the SDSS
spectroscopic object catalog during the SQLS?

\item What challenges will be faced when attempting a similar computation in a
wide-field imaging survey covering thousands of square degrees and potentially
millions of targets? And with which further approximations might these be
approached?

\end{enumerate}



%%%%%%%%%%%%%%%%%%%%%%%%%%%%%%%%%%%%%%%%%%%%%%%%%%%%%%%%%%%%%%%%%%%%%%%%
%%  SECTION 2: MODELS AND INFERENCE
%%%%%%%%%%%%%%%%%%%%%%%%%%%%%%%%%%%%%%%%%%%%%%%%%%%%%%%%%%%%%%%%%%%%%%%%

\section{Models and Inference}
\label{sec:model}


%%%%%%%%%%%%%%%%%%%%%%%%%%%%%%%%%%%%%%%%%%%%%%%%%%%%%%%%%%%%%%%%%%%%%%%%
%%  SECTION 3: DATA
%%%%%%%%%%%%%%%%%%%%%%%%%%%%%%%%%%%%%%%%%%%%%%%%%%%%%%%%%%%%%%%%%%%%%%%%

\section{Test Dataset}
\label{sec:data}


%%%%%%%%%%%%%%%%%%%%%%%%%%%%%%%%%%%%%%%%%%%%%%%%%%%%%%%%%%%%%%%%%%%%%%%%
%%  SECTION 4: RESULTS
%%%%%%%%%%%%%%%%%%%%%%%%%%%%%%%%%%%%%%%%%%%%%%%%%%%%%%%%%%%%%%%%%%%%%%%%

\section{Results}
\label{sec:results}


% %%%%%%%%%%%%%%%%%%%%%%%%%%
% \begin{figure*}
% \centerline{
% \includegraphics[width=0.9\linewidth]{figs/sth.png}}
% \caption{}
% \label{fig:sth}
% \end{figure*}
% %%%%%%%%%%%%%%%%%%%%%%%%%%


%%%%%%%%%%%%%%%%%%%%%%%%%%%%%%%%%%%%%%%%%%%%%%%%%%%%%%%%%%%%%%%%%%%%%%%%
%%  SECTION 5: CONCLUSIONS
%%%%%%%%%%%%%%%%%%%%%%%%%%%%%%%%%%%%%%%%%%%%%%%%%%%%%%%%%%%%%%%%%%%%%%%%

\section{Conclusions}
\label{sec:concl}

Our main results can be summarised as follows:

\begin{enumerate}

\item We now understand this.

\item And that.

\end{enumerate}


% do these comment headers really help?
% after all, the \section{} commands say THE SAME THING
% PJM: Yes, they break up the text in the editor, making it easier to navigate
% through a long file.
%%%%%%%%%%%%%%%%%%%%%%%%%%%%%%%%%%%%%%%%%%%%%%%%%%%%%%%%%%%%%%%%%%%%%%%%
%%  ACKNOWLEDGMENTS
%%%%%%%%%%%%%%%%%%%%%%%%%%%%%%%%%%%%%%%%%%%%%%%%%%%%%%%%%%%%%%%%%%%%%%%%

\section*{Acknowledgments}
 
We thank Sherry Suyu and James H.\ H.\ Chan, and Matt Auger and the STRIDES
collaboration for useful discussions and suggestions.
PJM acknowledges support from grants\ldots
DWH acknowledges support from grants\ldots

PJM was given support by the Royal 
Society, in the form of a research fellowship.
%
DWH ...
%
DL ...
%
% PS1



%%%%%%%%%%%%%%%%%%%%%%%%%%%%%%%%%%%%%%%%%%%%%%%%%%%%%%%%%%%%%%%%%%%%%%%%
%%  REFERENCES
%%%%%%%%%%%%%%%%%%%%%%%%%%%%%%%%%%%%%%%%%%%%%%%%%%%%%%%%%%%%%%%%%%%%%%%%

% MNRAS does not use bibtex, input .bbl file instead. 
% Generate this in the makefile using bubble script in scriptutils:

%   bubble -f swells-survey.tex references.bib 

\bibliographystyle{apj}
\bibliography{references}

%%%%%%%%%%%%%%%%%%%%%%%%%%%%%%%%%%%%%%%%%%%%%%%%%%%%%%%%%%%%%%%%%%%%%%%%

\label{lastpage}
\bsp

\end{document}

%%%%%%%%%%%%%%%%%%%%%%%%%%%%%%%%%%%%%%%%%%%%%%%%%%%%%%%%%%%%%%%%%%%%%%%%
